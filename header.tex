% 発音記号
\usepackage{tipa}

% 既定をゴシック体に
\renewcommand{\kanjifamilydefault}{\gtdefault}

% ヘッダを消す
% \beamertemplatenavigationsymbolsempty
% \setbeamertemplate{headline}{}

% フットラインにフレーム番号を挿入
\setbeamertemplate{footline}[frame number]
% 大きさを\normalsizeに変更,色をgrayに変更
\setbeamerfont{page number in head/foot}{size=\scriptsize}
\setbeamercolor{page number in head/foot}{fg=gray}

% listings
\lstset{
    frame=none,         % 枠設定
    breaklines=true,    % 行が長くなった場合自動改行
    breakindent=12pt,   % 自動改行時のインデント
    columns=fixed,      % 文字の間隔を統一
    basewidth=0.5em,    % 文字の横のサイズを小さく
    % numbers=left,       % 行数の位置
    % numberstyle={\tiny \color{white}},  % 行数のフォント
    % stepnumber=1,       % 行数の増間
    % numbersep=1\zw,      % 行数の余白   
    xleftmargin=0\zw,    % 左の余白
    xrightmargin=0\zw,   % 右の余白
    aboveskip=0.5\zw,      % 上の余白
    belowskip=0.5\zw,      % 下の余白
    framexleftmargin=0.5\zw,  % フレームからの左の余白
    keepspaces=true,    % スペースを省略せず保持
    lineskip=-0.3ex,    % 枠線の途切れ防止
    tabsize = 4,        % タブ数
    showstringspaces=false,  %文字列中の半角スペースを表示させない
    %%%%% VSCode風 の style & color %%%%%
    backgroundcolor={\color[gray]{0.1}},                    %背景色
    basicstyle     ={\small\ttfamily \color{white}},        % 基礎の文字のフォント設定
}

% インラインコードの設定
\usepackage{xpatch}
\usepackage{xcolor}
\usepackage{realboxes}
\definecolor{mygray}{rgb}{0.1,0.1,0.1}

\makeatletter
\xpretocmd\lstinline{\Colorbox{mygray}\bgroup\appto\lst@DeInit{\egroup}}{}{}
\makeatother